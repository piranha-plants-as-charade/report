\renewcommand{\maketitlehookd}{%
	\begin{abstract}
		\noindent
        The premise of ``Piranha Plants as Charade'' is to transform a melody into a full-fledged song in the style of ``Piranha Plants on Parade'' from the video game ``Super Mario Bros. Wonder''. Given an input melody in the form of a digital signal (e.g. a WAV file), we developed a music generation algorithm that transforms the input with the following process: 1) it extracts the melodic information from the input; 2) it generates a chord progression that fits under both the extracted melody and the appropriate stylistic conventions; 3) it generates a musical accompaniment based on the melody and chord progression; and 4) it exports the generated accompaniment as a WAV file. For inputs within the targeted scope (i.e. in 4/4, in C major, and at 110 BPM), our algorithm can often produce convincing results, but for a majority of the cases, the output is subpar. Step 1 (melody extraction) is most likely the largest source of failure; the pitch detection is generally correct, however, the onset detection often produces false positives. Despite the current shortcomings, we believe that the core ideas behind ``Piranha Plants as Charade'' can be extended upon to adequately meet the premise we set. We plan on fine-tuning our algorithm and further iterating over our code to produce better results.
	\end{abstract}
}