\subsection{Avenues for Future Work}
\label{sec:avenues}

\subsubsection{Relax MVP Assumptions}

The current implementation of Piranha Plants as Charade is a proof-of-concept that demonstrates the feasibility of our approach. However, it is limited by the assumptions we made to simplify the problem. Thus, we would like to relax these assumptions to expand the range of melodies that can be processed.

For example, we assume that the input melody is played at 110 BPM, which is the tempo of the original song. Moving forward, we would like to step away from this assumption and allow for arbitrary tempos. This would require us to implement a more sophisticated melody extraction algorithm that can handle varying tempos. There are existing methods to identify the beat and infer the tempo, such as BeatNet \cite{BeatNet:2021}.

We also assume that the key is C major to simplify the chord generation process. Although any melody can be transposed\footnote{To change the key of a piece of music by shifting all the notes up or down by a constant interval, maintaining the original contour and harmonic quality.} to match this key, as we have done for some sample inputs in this paper, we would ideally like to support any melody in its original key. There are various methods to solve the key-finding problem, such as the Krumhansl-Schmuckler key-finding algorithm \autocite{KrumhanslSchmuckler:1992}, an extension of Krumhansl-Schmuckler using intervals \autocite{MadsenWidmer:2007}, and more modern approaches using supervised learning \autocite{Mahieu:2016}.

\subsubsection{Improving Melody Extraction}

Given a simple synthetic melody within our MVP scope, our current implementation of melody extraction works sufficiently well; however, it is not robust enough for recorded audio data. In particular, it does not handle noise well and often has trouble accurately identifying articulations in the melody. Since our onset detection implementation seems to be the source of the most errors, a key point for future work is to refine this step of the pipeline.

\subsubsection{Expanding to Additional Styles}

In the current implementation, the chord progression we generate is determined by the observation and transition models that we define by hand. These capture the cohesiveness of a melody and a chord, and preferences for how to transition between chords respectively. For this MVP, we have only implemented the style of ``Piranha Plants on Parade'' by specifying a particular instance of the transition and observation models. However, we can easily expand this to other styles by defining new transition and observation models according to different harmonic patterns.

Unfortunately, defining the transition and observation models by hand is a tedious process that requires a deep understanding of music theory. A more scalable method would be to use a data-driven approach to learn the transition and observation models from a corpus of music. This would allow us to tailor the models to a specific style of music programmatically. It is worth noting that a bottleneck for this approach is the availability of data, as it would require a melody and the corresponding chord progression aligned with it.

\subsubsection{Extensions of the Hidden Markov Model}

One key assumption for the first-order hidden Markov model is the Markov assumption: that the next state depends only on the current state, and not the past \autocite{SpeechLang:2025}. While this assumption simplified our approach and provided a simple solution to solve it, this assumption limits the model's ability to capture long-term dependencies, which could be an important element in generating more complex chord progressions. For example, one common cadence\footnote{A harmonic progression signalling the end of a phrase, often creating a sense of resolution.} appearing in ``Piranha Plants on Parade'' and across many other genres, is the ii-V-I cadence. To encourage this cadence, we would need to define a transition model that captures the transitions between three chords. However, this is not possible with the first-order HMM, as its memory is limited to the most recent state. Higher-order HMMs would be able to capture patterns spanning multiple chords. Algorithms for solving higher-order HMMS have been developed, such as the Baum-Welch algorithm \autocite{BaumWelch:1970}, a method by Ye and Wang that entails encoding previous states as tuples \autocite{DecodeHMM:2014}, and an extended Viterbi algorithm for second-order HMMs \autocite{Viterbi2:1988}.

\subsubsection{Real-Time Processing}

One interesting expansion of Piranha Plants as Charade is to implement real-time processing of audio input. This would allow for a more interactive freestyle experience, where users can sing or play along with a generated harmony. However, one fundamental flaw of our current approach is that our chord generation algorithm assumes we have access to the full melody to predict the entire chord sequence at once. This is not the case in real-time processing, where we only have access to the melody up to the current time step. Thus, we would need to pivot to a different approach to chord generation that can predict the next chord based on the current melody, and it must be fast enough to do so in real time.

One approach building upon our current work is to process each time step independently, applying the transition and observation models once to determine the next chord. However, this greedy approach would not be able to capture the long-term dependencies of the melody.

We also had the opportunity to discuss our project with Professor Kate Larson\footnote{\url{https://cs.uwaterloo.ca/~klarson/}}, who mentioned a recent paper using reinforcement learning for real-time interactive jamming \autocite{ReaLJam:2025}. Although this is outside the scope of our project and knowledge to discuss in detail, it would be an interesting avenue to explore as we consider real-time processing.

% \subsubsection{Optimizing MIDI to WAV Conversion}
