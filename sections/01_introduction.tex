\section{Introduction}

The fields of mathematics and music are heavily intertwined. In the late 16th-century, many Western European music theorists believed that they had developed \emph{ars perfecta}: a set of rules for which, if followed, guaranteed that music be ``free of reprehensible elements, purged of every error and polished, and [the] harmonies will be good and pleasant'' \autocite{Richard:2005}. Although the concept of \emph{ars perfecta} is irrelevant in modern times, we sympathize with their aspirations to model musical correctness with rule-based approaches. We wanted to explore the relationship between music and mathematics in a similar manner by generating music using rule-based algorithms that humans follow --- processes that mirror the human composer. Our goal was to create a computer program that transforms a melody into a full-fledged song in the style of ``Piranha Plants on Parade'', a song from the video game \emph{Super Mario Bros. Wonder}.

We believe that ``Piranha Plants on Parade'' was an apt song to emulate for our proof-of-concept. It is based on a relatively simple harmonic framework, which lets us focus on developing the algorithm's high-level concepts rather than on hard-coding harmonic rules; yet the transitions between harmonic states\footnote{The chord changes.} are distinct enough to be recognizable. Similarly, the song's accompaniment\footnote{The music besides the melody.} style is repetitive enough such that it can be approximated with only a few rules. ``Piranha Plants on Parade'' also distinctively features gibberish lyrics, which lets us explore vocal generation without worrying about conforming to an existing language. Lastly, the song originates from a video game, a medium with a history of synthetic music, thus we believe it to be a thematically appropriate subject for computer generated music.
