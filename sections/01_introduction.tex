\section{Introduction}

The fields of mathematics and music are heavily intertwined. In the late 16th-century, many Western European music theorists believed that they had developed \emph{ars perfecta}: a set of rules for which, if followed, guaranteed that music be ``free of reprehensible elements, purged of every error and polished, and [the] harmonies will be good and pleasant'' \autocite{Richard:2005}. Although the concept of \emph{ars perfecta} is a footnote in the modern musical landscape, we sympathize with their aspirations to model musical correctness with rule-based approaches. We wanted to similarly explore the relationship between music and mathematics by generating music using rule-based algorithms that humans follow --- processes that mirror the human composer. Our goal was to create a proof-of-concept computer program that transforms a melody into a full-fledged song in the style of ``Piranha Plants on Parade''\footnote{Link: \href{https://www.youtube.com/watch?v=3EkzTUPoWMU}{www.youtube.com/watch?v=3EkzTUPoWMU}.} from the video game \emph{Super Mario Bros. Wonder}.

We believe that ``Piranha Plants on Parade'' was the ideal song to emulate for our proof-of-concept. It is based on a relatively simple harmonic framework, which lets us focus on developing the algorithm's high-level concepts rather than on hard-coding harmonic rules; yet the transitions between harmonic states are distinct enough to be recognizable. Similarly, the song's accompaniment style is repetitive enough such that it can be approximated with only a few rules. ``Piranha Plants on Parade'' also distinctively features gibberish lyrics, which lets us explore vocal generation without worrying about conforming to an existing language. Lastly, the song originates from a video game, a medium with a long history of using synthetic audio and music, thus we believe it to be a thematically appropriate subject for computer-generated music.

We worked on Piranha Plants as Charade for close to two months --- from February 10 to March 24, 2025 --- and we borrowed ideas from several areas of computer science and music, including but not limited to: digital signal processing (DSP), hidden Markov models (HMM), and Western music theory. Due to time and budget limitations and our lack of expertise across our explored domains, we believed it to be unreasonable for us to perfectly emulate the style of ``Piranha Plants on Parade''. Thus, our goal was to create a minimum viable product (MVP), which we defined as follows:
\begin{enumerate}
    \item Our MVP should support melody inputs with the following properties:
    \begin{enumerate}
        \item Has a time signature of 4/4.
        \item Is in the key of C major.
        \item Has a tempo of 110 beats per minute.
        \item Starts on a note (i.e. not a rest).
    \end{enumerate}
    \item For any supported input, our MVP should output a song with the following parts:
    \begin{enumerate}
        \item The input melody and a harmony line, both sung in a `Piranha Plant style'.
        \item A stride piano part that outlines an appropriate chordal accompaniment.
    \end{enumerate}
\end{enumerate}
In terms of our MVP scope, we are satisfied with our observed performance of Piranha Plants as Charade. We were also able to implement non-MVP features such as the integration of basic percussion in the output and the live deployment of our project in the form of an application programming interface (API).