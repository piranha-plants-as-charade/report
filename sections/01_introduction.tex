\section{Introduction}

\subsection{Motivation}

The fields of mathematics and music are heavily intertwined. In the late 16th-century, many Western European music theorists believed that they had developed \emph{ars perfecta}: a set of rules for which, if followed, guaranteed that music be ``free of reprehensible elements, purged of every error and polished, and [the] harmonies will be good and pleasant'' \autocite{Richard:2005}. Although the concept of perfect music is a footnote in the modern musical landscape, we are intrigued by their aspirations to model musical correctness with rule-based approaches. We wanted to similarly explore the relationship between music and mathematics by generating music using rule-based algorithms that humans follow --- processes that mirror the human composer. Our goal was to create a proof-of-concept computer program that transforms a melody into a full-fledged song in the style of ``Piranha Plants on Parade''\footnote{\href{https://www.youtube.com/watch?v=3EkzTUPoWMU}{youtube.com/watch?v=3EkzTUPoWMU}} from the video game \emph{Super Mario Bros. Wonder}.

\subsubsection{Why ``Piranha Plants on parade''?}

We believe that this was the ideal song to emulate for our proof-of-concept. It is based on a relatively simple harmonic framework, which lets us focus on developing the algorithm's high-level concepts rather than on hard-coding harmonic rules; yet the transitions between harmonic states are distinct enough to be recognizable. Similarly, the song's accompaniment style is repetitive enough such that it can be approximated with only a few rules. ``Piranha Plants on Parade'' also distinctively features gibberish lyrics, which lets us explore vocal generation without worrying about conforming to an existing language. Lastly, the song originates from a video game, a medium with a long history of using synthetic audio and music, thus we believe it to be a thematically appropriate subject for computer-generated music.

\subsection{Minimum Viable Product (MVP)}

We worked on Piranha Plants as Charade for one-and-a-half months --- from February 10 to March 24, 2025 --- and we borrowed ideas from several areas of computer science and music, including but not limited to: digital signal processing (DSP), hidden Markov models (HMM), Western music theory, and Jazz performance. Due to time and budget limitations and our lack of expertise across our explored domains, we believed that it was unreasonable for us to perfectly emulate the style of ``Piranha Plants on Parade''. Thus, our goal was to create an MVP, which we defined as follows:
\begin{enumerate}
    \item The MVP should support melody inputs with the following properties:
    \begin{enumerate}
        \item Has a time signature of 4/4.
        \item Is in the key of C major.
        \item Has a tempo of 110 beats per minute.
        \item Starts on a note (i.e. not a rest).
    \end{enumerate}
    \item For any supported input, the MVP should output a song with the following parts:
    \begin{enumerate}
        \item The input melody and a harmony line, both sung in a `Piranha Plant style'.
        \item A stride piano part that outlines an appropriate chordal accompaniment.
    \end{enumerate}
\end{enumerate}
We consider Piranha Plants as Charade to have adequately met our MVP goals. We also implemented additional features, including exported percussion parts and live deployment in the form of an application programming interface (API).