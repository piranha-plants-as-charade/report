\subsection{Chord Generation}
\label{sec:chord_generation}

The next step in Piranha Plants as Charade is to generate the `best' chord progression\footnote{A sequence of chords.} for the extracted melody from Part \ref{sec:melody_extraction}. It is generally quite open-ended and subjective whether a chord progression `sounds good' with a melody, and there can be many different appropriate chord progressions that evoke different emotions and styles. For this project, we define `appropriate` to mean in adherance with the style of Piranha Plants. We use a first-order Hidden Markov Model (HMM) framework to determine the most appropriate chord progression.

The Hidden Markov Model \autocite{HMM:2023,SpeechLang:2025}, depicted in Figure \ref{fig:hmm}, is a statistical model that describes a sequence of hidden states, each of which emits an observation. In the context of chord generation, we model the hidden states as the sequence of chords and the observations as a function of the melody. The HMM framework allows us to model the probability of a chord sequence given a melody, which we can then use to generate the most likely or most favorable sequence of chords.

The model consists of the following components for time step $t \geq 0$:
\begin{itemize}
    \item \textbf{States} $s_t$: The hidden states of the model, which represent the sequence of chords.
    \item \textbf{Observations} $o_t$: The observations of the model, which represent the melody.
    \item \textbf{Initial State Probabilities (Priors)} $P(s_0)$: The probabilities of starting in a particular state.
    \item \textbf{Transition Probabilities} $P(s_{t+1} | s_t)$: The probabilities of transitioning between states.
    \item \textbf{Observation Probabilities} $P(o_t | s_t)$: The probabilities of emitting an observation given a state.
\end{itemize}

\begin{figure}
    \resizebox{\linewidth}{!}{\begin{tikzpicture}[->, >=stealth, node distance=1cm, auto, scale=1, transform shape]
    % Piranha Plants score snippet
    \node [inner sep=0pt] (notes) {
        \includegraphics[width=7cm]{figures/piranha_plants_snippet_.png}
    };

    \node[draw, red, thick, anchor=north west, minimum width=1.7cm, minimum height=1.3cm] (notes1) at ([xshift=1.25cm, yshift=-0.1cm] notes.north west) {};
    \node[draw, blue, thick, anchor=north west, minimum width=1.6cm, minimum height=1.3cm] (notes2) at ([xshift=3cm, yshift=-0.1cm] notes.north west) {};

    % Hidden states
    \node[draw, circle, below=of notes.west, anchor=west, xshift=0.3cm, yshift=-2.5cm] (s0) {$s_0$};
    \node[draw, circle, right=of s0] (s1) {$s_1$};
    \node[draw, circle, right=of s1] (s2) {$s_2$};
    \node[right=0.5cm of s2] (dots) {\Large $\dots$};
    \node[draw, circle, right=0.3cm of dots] (st) {$s_t$};
    
    % Observation states
    \node[draw, circle, red, above=of s1] (o1) {$o_1$};
    \node[draw, circle, blue, above=of s2] (o2) {$o_2$};
    \node[draw, circle, above=of st] (ot) {$o_t$};

    % Lines from notes to observation nodes
    \draw[red, -] ($(notes1.south west) + (0.01, 0)$) -- (o1.west);
    \draw[red, -] ($(notes1.south east) - (0.01, 0)$) -- (o1.east);
    \draw[blue, -] ($(notes2.south west) + (0.01, 0)$) -- (o2.west);
    \draw[blue, -] ($(notes2.south east) - (0.01, 0)$) -- (o2.east);
    
    % Hidden state transitions
    \draw[->] (s0) -- (s1);
    \draw[->] (s1) -- (s2);
    \draw[->] (s2) -- (dots);
    \draw[->] (dots) -- (st);
    
    % Observation emissions
    \draw[->] (o1) -- (s1);
    \draw[->] (o2) -- (s2);
    \draw[->] (ot) -- (st);
\end{tikzpicture}
}
    \caption{Hidden Markov Model for chord generation.}
    \label{fig:hmm}
\end{figure}

In the most basic case, states and observations can be mapped to integer indices such that the transition and observation probabilities can be represented as matrices. The model can then be solved using the Viterbi \autocite{HMM:2023} algorithm, which is a dynamic programming algorithm that finds the most likely sequence of hidden states given the observations.

For this project, we implemented the transition and observation probabilities as scores (i.e. unnormalized, relative probabilities). The transition scores represent the favorability of transitioning between two chords, while the observation scores represent the favorability of a chord given the melody.

\subsubsection{Transition Scores}

Our transitions are based on an analysis of ``Piranha Plants on Parade'' through a Western music theory perspective. We identified the following patterns:
\begin{enumerate}
    \item The I, IV, and V chords\footnote{TODO: explain this} and their respective secondary dominant chords\footnote{TODO explain this} almost exclusively used.
    \item A secondary dominant chord must be followed by their respective I, IV, or V chord.
    \item Secondary dominant chords occur frequently.
\end{enumerate}
In our transition scores, we restricted the valid states to follow the first two patterns, and we applied a 1.5x score multiplier for transitioning away from a secondary dominant chord to model the last pattern.

\subsubsection{Observation Scores}

The observation scores are based on the melody. We calculate the score of a melody given a chord by summing the number of notes in the melody that are in the chord. This is a simple heuristic that favors chords that contain more notes from the melody, which is generally a good rule of thumb for harmonization. Although this doesn't fully account for deviations such as passing tones, it provides a good starting point for the model.
